\section{Grundbegriffe allg. Topologie}

\paragraph{Topologische Räume}
\begin{itemize}
  \item \textbf{Topologie}: \( \O \subseteq \mathcal{P}(X) \) (Menge \( X \)) sodass
  \begin{itemize}
    \item \( X, \varnothing \in \O \)
    \item Durchschnitte \emph{endlich} \& Vereinigungen \emph{beliebig} vieler Mengen aus \( \O \) in \( \O \)
  \end{itemize}
  \item \textbf{Topologischer Raum}: \( \left( X, \O \right) \)
  \begin{itemize}
    \item \emph{Offene Teilmengen} von \( X \): Elemente von \( \O \)
    \item \emph{Abgeschlossene Teilmengen} \( A \subset X \): \( X \setminus A \) offen
  \end{itemize}
  \item \textbf{Wichtige Topologien}:
  \begin{itemize}
    \item \emph{Triviale Topologie}: \( \O_{\text{trivial}} \coloneqq \left \{ X, \varnothing \right \} \)
    \item \emph{Diskrete Topologie}: \( \O_{\text{diskret}} \coloneqq \mathcal{P}(X) \)
    \item \emph{Standard-Topologie} auf \( \R \): \( \O_s \coloneqq \left \{ I \subset \R : I = \text{ Vereinigung offener Intervalle} \right \} \)
    \item \emph{Zariski-Topologie}: \( \O_Z \coloneqq \left \{ O \subset \R : O = \R \setminus E\text{, } E \subset \R \text{ endlich} \right \} \cup \left \{ \varnothing \right \} \)
    \item \emph{Induzierte Topologie} (Metrik):
    \begin{itemize}
      \item \( U \subset X \) \emph{d-offen} \( \Leftrightarrow \forall p \in U \ \exists \epsilon = \epsilon(p) > 0 : B_\epsilon(p) \subset U \)
      \item \( d \)-offene Mengen bilden \emph{induzierte Topologie}
    \end{itemize}
    \item \emph{Teilraum-Topologie}: \( \O_Y \coloneqq \left \{ U \subseteq Y : \ \exists \ V \in \O_X : U = V \cap Y \right \} \) \\ (Topologischer Raum \( (X, \O_X) \), Teilmenge \( Y \subseteq X \))
    \item \emph{Produkttopologie}: Topologische Räume \( (X, \O_X) \), \( (Y, \O_Y) \) \\ \( W \subseteq X \times Y \) \emph{offen} in \emph{Produkttopologie} \( \Leftrightarrow \forall (x,y) \in W \ \exists \) Umgebung \( U \) von \( x \) in \( X \) und \( V \) von \( y \) in \( Y \), sodass \( U \times V \subseteq W \)
    \item \emph{Quotiententopologie}: \( (X, \O) \) topologischer Raum, \( \pi : X \ni x \mapsto [x] \in X/\sim \) kanonische Projektion
    \begin{itemize}
      \item[\( \to \)] \( U \subset X/\sim \) ist offen \( \Leftrightarrow \pi^{-1}(U) \) ist offen in \( X \).
    \end{itemize}
  \end{itemize}
  \item \textbf{Basis} für Topologie \( \O \): \( \mathcal{B} \subset \O \) sodass für jede offene Menge \( \varnothing \neq V \in \O \) gilt
  \begin{equation*}
    V = \bigcup_{i \in I} V_i, \quad V_i \in \mathcal{B}
  \end{equation*}
  \item \textbf{Umgebung} \( U \subset X \) von \( A \subset X \), falls \( \exists \ O \in \O : A \subset O \subset U \) \\
  (Topologischer Raum \( (X, \O) \))
  \item \textbf{Innerer, äußerer Punkt} \( p \in X \) von \( A \subset X \), falls \( A \) (bzw. \( X \setminus A \)) Umgebung von \( \left \{ p \right \} \) ist
  \begin{itemize}
    \item[\( \to \)] \emph{Inneres} von \( A \subset X \): Menge \( \mathring{A} \) der inneren Punkte von \( A \)
  \end{itemize}
  \item \textbf{Abgeschlossene Hülle} von \( A \): Menge \( \overline{A} \subset X \), die \emph{nicht} äußere Punkte sind
  \item \textbf{Triangulierbar}: falls \( \exists \) Simplizialkomplex \( K \) und Homö \( K \to X \)
  \begin{itemize}
    \item \( \chi(X) \coloneqq \chi(K) \)
  \end{itemize}
\end{itemize}

\paragraph{Hausdorffsches Trennungsaxiom}
\begin{itemize}
  \item \textbf{Hausdorffsch} (top. Raum \( (X, \O) \)): \( \forall p \neq q \in X \ \exists \  U \ni p, V \ni q : U \cap V = \varnothing \) (\emph{Umgebungen} \( U \), \( V \)) 
  \item \textbf{Hausdorffsche Räume}:
  \begin{itemize}
    \item Metrische Räume (über Dreiecks-Ugl.)
    \item \( (\R, \O_{\text{s}}) \), weil \( \O_{\text{s}} \) von Metrik induziert wird
    \item Teilraum von Hausdorff-Raum
    \item Produkt von Hausdorff-Räumen bzgl. Produkttopologie
  \end{itemize}
\end{itemize}

\paragraph{Stetigkeit}
\begin{itemize}
  \item \textbf{Stetigkeit} (zwischen top.\ Räumen \( (X,\O_X) \), \( (Y, \O_Y) \)): \( f: X \to Y \) stetig falls Urbilder offener Mengen in \( Y \) offen sind in \( X \)
  \item \textbf{Homöomorphismus} (zw.\ top.\ Räumen): \( f: X \to Y \) bijektiv mit \( f \), \( f^{-1} \) stetig
  \begin{itemize}
    \item[\( \to \)] \( X \), \( Y \) \emph{homöomorph}, falls \( \exists \) Homö \( f: X \to Y \) (schreibe \( X \cong Y \))
    \item \emph{Homöomorphismengruppe}: Identität, Verkettungen, Inverse von Homö sind Homö
    \begin{itemize}
      \item[\( \to \)] Gruppe
    \end{itemize}
  \end{itemize}
  \item \textbf{Wichtige Homöomorphismen}: 
  \begin{itemize}
    \item \( [0,1] \cong [a,b] \) (\( a < b \in \R \))
    \item \( S^n \setminus \left \{ \left( 0,\dots,0,1 \right) \right \} \cong \R^n \) (also \( S^n \) ohne ``Nordpol'')
  \end{itemize}
\end{itemize}

\paragraph{Zusammenhang}
\begin{itemize}
  \item \textbf{Definition}: \( (X,\O) \) zusammenhängend, falls \( \varnothing \) und \( X \) die einzigen offen-abgeschlossenen Teilmengen sind
  \begin{itemize}
    \item[\( \Leftrightarrow \)] \( X \) ist \emph{nicht} disjunkte Vereinigung von \( 2 \) offenen, nichtleeren Mengen
  \end{itemize}
  \item \textbf{Eigenschaften}:
  \begin{itemize}
    \item \( A \) zusammenhängend \( \Rightarrow \) \( \overline{A} \) ist zusammenhängend
    \item \( A \), \( B \) zusammenhängend, \( A \cap B \neq \varnothing \) \( \Rightarrow \) \( A \cup B \) zusammenhängend
  \end{itemize}  
\end{itemize}

\paragraph{Zusammenhangskomponente}
\begin{itemize}
  \item \textbf{Definition}: \( Z(x) = \) Vereinigung aller zusammenhängender Teilmengen, die \( x \) enthalten
  \item \textbf{Eigenschaften}:
  \begin{itemize}
    \item \( Z(X) =  \) disjunkte Zerlegung von \( X \)
    \item Elemente von \( Z(X) = \) zusammenhängend
  \end{itemize}
\end{itemize}

\paragraph{Weg-Zusammenhang}
\begin{itemize}
  \item \textbf{Definition}: \( (X, \O) \) weg-zusammenhängend
  \begin{equation*}
    \Leftrightarrow \forall p,q \in X \ \exists \text{ Weg } \alpha : [0,1] \to X : \alpha(0) = p \wedge \alpha(1) = q
  \end{equation*}
  \item \textbf{Eigenschaften}:
  \begin{itemize}
    \item \( X \) weg-zusammenhängend \( \Rightarrow \) \( X \) zusammenhängend
    \item Stetige Bilder von (weg-)zusammenhängenden Räumen sind es auch
    \item Ein (nicht) zusammenhängender Raum kann nur zu einem (nicht) zusammenhängenden Raum homöomorph sein
  \end{itemize}
\end{itemize}

\paragraph{Kompaktheit}
\begin{itemize}
  \item \textbf{Definition}: \( (X,\O) \) kompakt \( \Leftrightarrow \) jede offene \( X \)-Überdeckung besitzt \emph{endliche} Teilüberdeckung:
  \begin{equation*}
    X = \bigcup_{i \in I} U_i,\ U_i \text{ offen } \Rightarrow \exists i_1,\dots,i_k \in I: X = U_{i_1} \cup \cdots \cup U_{i_k}
  \end{equation*}
  \item \textbf{Lokal kompakt}: Jeder Punkt von \( X \) besitzt kompakte Umgebung
  \item \textbf{Eigenschaften}:
  \begin{itemize}
    \item Man kann von lokale auf globale Eigenschaften schließen
    \item[\( \to \)] \( X \) kompakt, \( f: X \to \R \) lokal beschränkt \( \Rightarrow \) \( f \) beschränkt
    \item Stetige Bilder kompakter Räume sind kompakt
    \item Abgeschlossene Teilräume kompakter Räume sind kompakt
    \item Produkte kompakter Räume sind kompakt
    \item Kompakte Mengen in Hausdorff-Räumen sind abgeschlossen
  \end{itemize}
\end{itemize}